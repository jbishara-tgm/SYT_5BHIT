\section*{Abstract}
\cfoot{Hannah Siegel}

Nowadays robotics has become one of the most crucial foundations of a day-to-day business in nearly every industrial environment. As a result of the cost-efficiency and the overall benefits of autonomous systems, robots are in high demand in this area and their technologies are put under constant improvement. It is the strong presence of robots which distinguishes factories in industrialized countries from their competitors in regions with low-cost production possibilities. Nevertheless in the field of mobile robotics, used mostly for logistic purposes, we have come to witness an extreme unused potential. Mobile systems are not always able to detect and therefore react to new changes or obstacles in their surrounding areas and this increases the risk of damaging material or – worse – of hurting humans.  Mostly these security concerns may lead to the fact that the path-planning possibilities of mobile robots are limited to only pre-defined and hardcoded pathways which excessively scaled down the flexibility of these systems.
The project - \textit{RoboNav} - aims to evaluate if a system with better positioning and more flexible navigation can be established if external image processing units deliver additional real-time information from a bird's eye perspective. 
In order to simulate these technical procedures in a school project, the overall idea had to be scaled down to simpler hardware and software environments while constantly trying to stick to the theoretical industrial environment and its challenges. The proof of concept comprises four entities: a mobile execution device, the \textit{Robotino® v3} from Festo AG \& co. KO, the external image capturing unit (\textit{RoboNav Sight}),  represented by an overhead Android smart phone with a streaming app, an intuitive graphical user interface (\textit{RoboNav Overview}) and in the center of the system the control application (\textit{RoboNav Control}), which contains the important aspects of \textit{RoboNav}. The responsibilities of \textit{RoboNav Control} are numerous: It acts as a communication middleware for all the other components, handles the image processing and information gathering structures and has to provide path-planning and path-driving functionalities. Additionally, a asynchronous communication link has been developed, \textit{RoboNav Communication Protocol} (\textit{RNCP}). The important part of improving the navigation is done by position comparison between the internal position of the robot's odometry and the correct position as seen from an overhead camera. The project is a proof of concept, but the outcome of this work can be easily transformed onto other systems with practically no changes to be done.
